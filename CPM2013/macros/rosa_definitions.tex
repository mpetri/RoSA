% O-notation
\newcommand{\Vzeropos}[0]{\ensuremath{\mathit{{zero\mathunderscore pos}}}} 
\newcommand{\Vones}[0]{\ensuremath{\mathit{{ones}}}} 
\newcommand{\Vdepth}[0]{\ensuremath{\mathit{{depth}}}}
\newcommand{\Vfactor}[0]{\ensuremath{\mathit{{factor}}}}
\newcommand{\Vcrank}[0]{\ensuremath{\mathit{c\mathunderscore rank}}}
\newcommand{\Vcpos}[0]{\ensuremath{\mathit{c\mathunderscore pos}}}
\newcommand{\firstrowcharacter}[1]{\mbox{\bf first\_row\_character\ensuremath{(\mbox{#1})}}}

\newcommand{\cmp}[1]{{\mbox{\tt{#1}}}}
\newcommand{\fwdbf}[0]{\ensuremath{\mathit{{bF}}}} % Bit vector that marks the left boundary of blocks in the forward index
\newcommand{\bwdbf}[0]{\ensuremath{\mathit{bf}}} % Bit vector that marks the left or right boundary in the backward index
\newcommand{\nbwdbf}[0]{\ensuremath{n_{\bwdbf}}} % Number of ones in \bwdbf.
\newcommand{\bwdbl}[0]{\ensuremath{\mathit{b\ell}}} % Bit vector holding the LF-permuted version of \bwdbf.
\newcommand{\bwdbm}[0]{\ensuremath{\mathit{bm}}} % Bit vector to map intervals in the backward index to blocks to the forward index.
\newcommand{\bwdmindepth}[0]{\ensuremath{\mathit{min\mathunderscore depth}}} % also used to map intervals in the bwd index to the fwd indexx
\newcommand{\deltax}[0]{\ensuremath{\Delta_{x}}} % Displacement of a reducible block in its corresponding irreducible block.
\newcommand{\deltad}[0]{\ensuremath{\Delta_{d}}} % Depth offset of a reducible block in its corresponding irreducible block.
\newcommand{\bwdpointer}[0]{\ensuremath{\mathit{pointer}}} % Array of pointers into the external structures (blocks or suffix array).

\newcommand{\threshold}{\ensuremath{b}}
\newcommand{\cB}[0]{\ensuremath{\mathit{k}}} % Number of blocks.
\newcommand{\cBSingleton}[0]{\ensuremath{\cB_s}} % Number of singleton blocks.
\newcommand{\cBReducible}[0]{\ensuremath{\cB_{\mathit{re}}}} % Number of reducible blocks.
\newcommand{\cBIrreducible}[0]{\ensuremath{\cB_{\mathit{ir}}}} % Number of irreducible blocks.
\newcommand{\cBfOnes}[0]{\ensuremath{\cB_{\bwdbf}}} % Number of ones in \bwdbf.
\newcommand{\bwdID}[0]{\ensuremath{\mathit{bwd\mathunderscore id}}} % Backward ID variable in the pseudo-code.
\newcommand{\bwdid}[1]{\ensuremath{\bwdID(#1)}} % Backward ID method.
\newcommand{\runnr}[0]{\ensuremath{run\mathunderscore nr}} % Number of run in bm.
\newcommand{\runpos}[0]{\ensuremath{run\mathunderscore pos}} % Position of a run in bm.
\newcommand{\blockaddr}[0]{\ensuremath{\mathit{block\mathunderscore addr}}} % Block address in the pseudo code.


% bit vector operations
\newcommand{\select}[0]{\ensuremath{\mathit{select}}}       
\newcommand{\rank}[0]{\ensuremath{\mathit{rank}}}    

% suffix tree operations
\newcommand{\sizeop}[0]{\ensuremath{\mathit{size}()}}      % size operation
\newcommand{\rootop}[0]{\ensuremath{\mathit{root}()}}      % root operation
\newcommand{\isleaf}[0]{\ensuremath{\mathit{is\mathunderscore leaf}}}
\newcommand{\isleafop}[0]{\ensuremath{\isleaf(v)}}      % root operation
\newcommand{\ithleaf}[0]{\ensuremath{\mathit{ith\mathunderscore leaf}}}      % returns the ith leaf in the CST
\newcommand{\ithleafop}[0]{\ensuremath{\ithleaf(i)}}      % returns the ith leaf in the CST

\newcommand{\nodesop}[0]{\ensuremath{\mathit{nodes}()}}	   % return the number of nodes in the ST
\newcommand{\idop}[0]{\ensuremath{\mathit{id}(v)}}         % returns an id in the range from
\newcommand{\parent}[0]{\ensuremath{\mathit{parent}}} % returns the parent of node v 
\newcommand{\parentop}[0]{\ensuremath{\parent(v)}} % returns the parent of node v 
                                       % or root() if v equals root()
\newcommand{\mydegree}[0]{\ensuremath{\mathit{degree}}}	
\newcommand{\degreeop}[0]{\ensuremath{\mydegree(v)}} % number of children of node v, note:
                                       % digree(root()) equals \alphabetsize
\newcommand{\childop}[0]{\ensuremath{\mathit{child}(v, c)}} % returns the child w of v whose edge
                                       % label of (v,w) start with symbol c,
									   % or root() if such w does not exists
\newcommand{\ithchild}[0]{\ensuremath{\mathit{ith\mathunderscore child}}}									   
\newcommand{\ithchildop}[0]{\ensuremath{\ithchild(v, i)}} % returns the ith child of v 
\newcommand{\firstchildop}[0]{\ensuremath{\ithchild(v, 1)}} % returns the ith child of v 
                                               %or root() if i>degree
\newcommand{\depth}[0]{\ensuremath{\mathit{depth}}}   
\newcommand{\depthop}[0]{\ensuremath{\depth(v)}}   % string depth of the concatenation of
                                       % all edge labels of the path from 
	                                   % root() to v
\newcommand{\nodedepthop}[0]{\ensuremath{\mathit{node\mathunderscore depth}(v)}} % Returns the depth of the node in the tree.									   
\newcommand{\edge}[0]{\ensuremath{\mathit{edge}}}  
\newcommand{\edgeop}[0]{\ensuremath{\edge(v, d)}}   % returns the d-th character (1-based 
		                               % indexing) of the edge-label pointing 
                                       % to v.
\newcommand{\lca}[0]{\ensuremath{\mathit{lca}}}									   
\newcommand{\lcaop}[0]{\ensuremath{\lca(v, w)}}     % returns the lowest common ancestor 
                                       % of v and w
\newcommand{\slop}[0]{\ensuremath{\mathit{sl}(v)}}         % suffix link operation: return the
                                       % node in the 
\newcommand{\wlop}[0]{\ensuremath{\mathit{wl}(v, c)}}       % Weiner link operation

\newcommand{\sibling}[0]{\ensuremath{\mathit{sibling}}} % returns the sibling of a node v
\newcommand{\siblingop}[0]{\ensuremath{\sibling(v)}} % returns the sibling of a node v
                                       % [0..nodes()-1] 
\newcommand{\nodeop}[0]{\ensuremath{\mathit{node}(\mathit{lb}, \mathit{rb})}} % Translate a lcp-interval to a
                                            % node in the ST
\newcommand{\lb}[0]{\ensuremath{\mathit{lb}}}         % Left bound of the lcp-interval of
\newcommand{\lbop}[0]{\ensuremath{\lb(v)}}         % Left bound of the lcp-interval of
                                       % node v
\newcommand{\rb}[0]{\ensuremath{\mathit{rb}}}         % Left bound of the lcp-interval of
\newcommand{\rbop}[0]{\ensuremath{\rb(v)}}         % Left bound of the lcp-interval of
                                       % node v
\newcommand{\poidx}{\ensuremath{\mathit{po\mathunderscore idx}}}     % Calculate the postorder index of the
\newcommand{\poidxop}{\ensuremath{\poidx(v)}}     % Calculate the postorder index of the
                                       % node v in the ST
\newcommand{\tlcpidx}{\ensuremath{\mathit{tlcp\mathunderscore idx}}}  % po_idx(lca(ith_leaf(i+1),ith_leaf(i+2)))
\newcommand{\tlcpidxop}{\ensuremath{\tlcpidx(i)}}  % po_idx(lca(ith_leaf(i+1),ith_leaf(i+2)))
\newcommand{\csaop}{\ensuremath{\mathit{csa}[i]}}	
\newcommand{\lcpop}{\ensuremath{\mathit{lcp}[i]}}	

\newcommand{\leavesinthesubtreeop}[0]{\ensuremath{\mathit{leaves\mathunderscore in\mathunderscore the\mathunderscore subtree}(v)}}									   
\newcommand{\rightmostleafinthesubtreeop}[0]{\ensuremath{\mathit{rightmost\mathunderscore leaf\mathunderscore in\mathunderscore the\mathunderscore subtree}(v)}}
\newcommand{\leftmostleafinthesubtreeop}[0]{\ensuremath{\mathit{leftmost\mathunderscore leaf\mathunderscore in\mathunderscore the\mathunderscore subtree}(v)}}	

% Algorithm notations
\newcommand{\Keyw}[1]{{\bf #1}}

% Arrays 
\newcommand{\BWT}[0]{\ensuremath{\mathsf{L}}}
\newcommand{\cBWT}[0]{\ensuremath{\BWT^{\TEXT^{r}}}}
\newcommand{\cL}[0]{\ensuremath{\mathsf{cL}}}
\newcommand{\cC}[0]{\ensuremath{\mathsf{cC}}}
\newcommand{\wt}[0]{\ensuremath{\mathit{wt}}}
\newcommand{\lcp}[0]{\ensuremath{\mathit{lcp}}}
\newcommand{\LCP}[0]{\ensuremath{\mathsf{LCP}}}
\newcommand{\SUF}[0]{\ensuremath{\mathsf{SA}}}
\newcommand{\ISA}[0]{\ensuremath{\mathsf{ISA}}}
\newcommand{\CSA}[0]{\ensuremath{\mathsf{CSA}}}
\newcommand{\CST}[0]{\ensuremath{\mathsf{CST}}}
\newcommand{\ST}[0]{\ensuremath{\mathsf{ST}}}
\newcommand{\LF}[0]{\ensuremath{\mathsf{LF}}}
\newcommand{\PSI}[0]{\ensuremath{\Psi}}
\newcommand{\CARRAY}[0]{\ensuremath{\mathsf{C}}}
\newcommand{\rankbwt}[0]{\ensuremath{\mathit{rank\mathunderscore bwt}}}
\newcommand{\selectbwt}[0]{\ensuremath{\mathit{select\mathunderscore bwt}}}
\newcommand{\TEXT}[0]{\ensuremath{\mathsf{T}}}
\newcommand{\PATT}[0]{\ensuremath{\mathsf{P}}}
\newcommand{\FROW}[0]{\ensuremath{\mathsf{F}}} % first row
\newcommand{\sentinel}[0]{\$}

% Big and small O-notation
\newcommand{\Order}[1]{\ensuremath{\mathcal{O}(#1)}}
\newcommand{\order}[1]{\ensuremath{o(#1)}}

\newcommand{\bsstate}[4]{(#1,#2)[#3..#4]} % backward search state
